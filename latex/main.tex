\documentclass[MScThesis,twoside]{lambek}

\def\enSubject{Doctor of Philosophy Thesis (Information Technology Major), Computer Engineering Department, Sharif University of Technology, Tehran, I. R. Iran}

\def\enTitle{Lambek Calculus and its Applications to German and Persian Languages}
\def\enTitleLineOne{Lambek Calculus with Applications}
\def\enTitleLineTwo{to German and Persian Languages}
\def\enAuthor{‌Muhammad-Ali A'rabi}
\def\enKeywords{Pregroup grammar, Lambek calculus, Machine translation}


%\input{general/preamble}
%\addbibresource{resources/resources.bib}


\newcommand{\faKeywords}{حساب لمبک،
زبان‌شناسی محض،
ترجمه‌ی ماشینی}
\logo{\includegraphics[scale=.7]{img/logo}}
\date{آبان ۱۳۹۳}
\titlelineone{حساب لمبک و کاربردهایش}
\titlelinetwo{در زبان‌های آلمانی و پارسی}
\title{حساب لمبک و کاربردهایش در زبان‌های آلمانی و پارسی}
\author{محمدعلی اعرابی}
\university{{\Large دانشگاه صنعتی شریف %
\\}
دانشکده‌ی علوم ریاضی}
\normaluniversity{دانشگاه صنعتی شریف\\
دانشکده‌ی علوم ریاضی}
\subject{گرایش منطق}
\supervisor{دکتر محمد اردشیر بهرستاقی}
\consult{دکتر بهرام وزیرنژاد} 

\newcommand{\efootnote}[1]{\footnote{\lr{#1}}}
\newcommand{\ecfootnote}[1]{}


% ================ Correct hyphenations ================
\hyphenation{test}


\makeglossaries
%\includeonly{related_works/related_works}
%\includeonly{evaluation/evaluation}

\begin{document}


\newcommand{\StartDocument}{\frontmatter \baselineskip1.2\baselineskip \pagestyle{empty} \null \vfill
\begin{center}
\Large{بسم اللّه الرحمن الرحیم}
\end{center}
\vfill}

%the initial title is supposed to be printed on the cover.
%for non final version, you can leave following commands as is to create only one title page (printed on paper)
%for final version you need to swap folowing commented/uncommented makethesistitle commands to achieve this order: title on the cover THEN in the name of god page THEN another title page but this time printed on paper
\makethesistitle
\StartDocument
%\makethesistitle

\pagestyle{pagenumberonlyPagestyle}
% following parts are not required in PhD proposal and should be removed. BEGIN OF COMMENT FOR PhD Proposal........
%صفحه‌ی تصویب در پیشنهاد پژوهشی وجود ندارد.
\begin{تصویب}
%خط‌های زیر در صورت نبود استاد مدعو comment شوند
\داور{استاد مدعو}{دکتر <نام استاد مدعو ۱>}
\داور{استاد مدعو}{دکتر <نام استاد مدعو ۲>}
\end{تصویب}
\newpage

\dedication{\large پیش‌کش به روان ژواخیم لمبک\\و میرشمس‌الدین ادیب‌سلطانی} 

\setlength{\baselineskip}{0.9cm}

\begin{قدردانی}
صفحه‌ی قدردانی. این صفحه اختیاری بوده و می‌توانید آن را حذف کنید. برای این کار کافی است محیط قدردانی در پرونده‌ی تِک را حذف کنید. متداول است که در این صفحه از خانواده، استادها و همکارهای خود قدردانی نمایید.
\end{قدردانی}
% END OF COMMENT FOR PhD Proposal.

\begin{abstract}{\faKeywords}
% abstract ...
% write it at the end...
%persian abstract

در حساب لمبک ساختار جمله توسط یک نیم‌گروه مطالعه و تحلیل می‌شود. این حساب کاربردهای فراوانی در زبان‌شناسی دارد. از جمله تشخیص خوش‌ساختی یک جمله در زبان توسط ماشین. برای انجام عملیات جبری روی نیم‌گروه نام‌برده نگره‌ی برهان به یاری می‌آید و ساختاری منطقی برای فروکاستن دنباله‌ی جبری به دست می‌دهد.
\end{abstract}


\setlength{\baselineskip}{0.9cm}
\pagenumbering{tartibi}\tableofcontents\listoftables\listoffigures
%list of abbreviations may be added here...

\mainmatter
%\PrepareForMainContent
%	In the name of GOD

\chapter{پیش‌نیازها} % introduciton ...
\label{chap:intro}

\trans{حساب نحوی}{syntactic calculus}،
یا \trans{حساب لمبک}{Lambek calculus}
\cite{Lambek:lambek1958}
توسعه‌یي ست از دستگاه‌های قدیمی‌تر که توسط 
\name{آزیمیرز آیدوکیویچ}{Kazimierz Ajdukiewicz}
\cite{Adjukiewicz}
 و 
\name{یهوشوئا بارهیلل}{Yehoshua Bar-Hillel}
\cite{Bar-Hillel}
 ساخته شده بودند.
نام حساب نحوی از آیدوکیویچ است، و به این اشاره دارد که این حساب به بررسی ساختار نحوی زبان می‌پردازد.

\section{جبر آیدوکیویچ-بارهیلل}

چندتایی 
$\mathfrak{A} = (\mathcal{A}, \otimes, /, \backslash, \rightarrow)$
را در نظر گیرید. مجموعه‌ی 
$\mathcal{T}$
را کوچک‌ترین مجموعه‌یي بگیرید که 
$\mathcal{A}$
را در بر دارد، و تحت سه عمل 
$\otimes, /, \backslash$
بسته است. این سه عمل در واقع صوری اند، و عنصرهای مجموعه‌ی
$\mathcal{T}$
به طور نحوی به استقرا ساخته می‌شوند.
مجموعه‌ی 
$\mathcal{T}$
را با
$\langle \mathcal{A} \rangle$
نیز نشان می‌دهند.
مجموعه‌ی
$\mathcal{T}$
را \trans{مجموعه‌ی گونه‌ها}{set of types}، و هر 
$t \in \mathcal{T}$
را یک \trans{گونه}{type} می‌نامیم.
مجموعه‌ی
$\mathcal{A}$
را نیز مجموعه‌ی گونه‌های \trans{ناشکستنی}{atomic} می‌نامیم.

\begin{definition}
چندتایی
$\mathfrak{A}$
را جبر آیدوکیویچ-بارهیلل می‌نامیم هر گاه

\begin{enumerate}
\item
برای هر
$a, b, c \in \mathcal{T}$
بداریم
$(a \otimes b) \otimes c \leftrightarrow a \otimes (b \otimes c)$،

\item 
برای هر
$a, b, c \in \mathcal{T}$
بداریم
$a \otimes b \rightarrow c$
اگر و تنها اگر
$a \rightarrow c/b$
اگر و تنها اگر
$b \rightarrow a \backslash c$،
\item
عمل 
$\rightarrow$
بازتابی و ترایا باشد.
\end{enumerate}
\end{definition}

در تعریف بالا،
$a \leftrightarrow b$
معادل است با
$a \rightarrow b$
و
$b \rightarrow a$.

\begin{theorem}
\label{theorem:1-1}
برای هر
$a, b, c \in \mathcal{T}$
رابطه‌های زیر درست اند:
\begin{enumerate}
\item
(قانون آیدوکیویچ)
$(a / b) \otimes b \rightarrow a$،
\item 
(برافرازش گونه)
$b \rightarrow (a / b) \backslash a$،
\item
(طرفین-وسطین)
$(a / b) \otimes (b / c) \rightarrow a / c$،
\item
(قانون \name{گیچ}{Geach})
$a / b \rightarrow (a / c) / (b / c)$،
\item
(قانون \name{کری}{Curry})
$(c / b) / a \leftrightarrow c / (a \otimes b)$.
\end{enumerate}
\end{theorem}

\begin{proof}
از بازتابی بودن
$\rightarrow$
داریم
$a / b \rightarrow a / b$
و از ویژه‌گی دوم جبر آیدوکیویچ-بارهیلل داریم
$(a / b) \rightarrow a / b$،
اگر و تنها اگر
$(a / b) \otimes b \rightarrow a$.
با استفاده از برابر هم‌ارز دیگر در ویژه‌گی دوم نیز به
$b \rightarrow (a / b) \backslash a$
می‌رسیم.
پس دو رابطه‌ی نخست به همین ساده‌گی برقرار اند. 
\end{proof}

برای اثبات دیگر بخش‌ها نخست چند لم نیاز است.

\begin{lemma}
\label{lemma:1-2}
داریم 
$(a \backslash b) / c \leftrightarrow a \backslash (b / c)$.
\end{lemma}

\begin{proof}
چنان که انتظار داریم، در اثبات این لم باید از ویژه‌گی نخست جبرهای آیدکیویچ-بارهیلل استفاده کنیم.
پس استراتژی برهان ترادیساندن این دو عملگر به عملگرهای 
$\otimes$
است، و سپس بهره‌جویی از ویژه‌گی نخست. این کار به ساده‌گی انجام می‌پذیرد. \\
بنا بر بازتابی بودن 
$\rightarrow$
داریم
\[ a \backslash (b / c) \rightarrow a \backslash (b / c), \]
پس با شکستن عمل سمت راست داریم
\[ a \otimes (a \backslash (b / c)) \rightarrow b / c, \]
و با بار دیگر شکستن عمل سمت راست و ترادیساندن آن به 
$\otimes$
داریم
\[ (a \otimes (a \backslash (b / c))) \otimes c \rightarrow b. \]
اکنون از ویژه‌گی نخست داریم
\[ a \otimes ((a \backslash (b / c)) \otimes c) \rightarrow (a \otimes (a \backslash (b / c))) \otimes c, \]
و از ترایا بودن 
$\rightarrow$
و آمیختن دو رابطه‌ی پیشین به دست می‌آوریم
\[ a \otimes ((a \backslash (b / c)) \otimes c) \rightarrow b, \]
با ترادیساندن 
$\otimes$
بیرونی به 
$\backslash$
می‌رسیم به
\[ (a \backslash (b / c)) \otimes c \rightarrow a \backslash b. \]
ترادیساندن عمل
$\otimes$
بیرونی این رابطه به 
$/$
رابطه‌ی 
\[ a \backslash (b / c) \rightarrow (a \backslash b) / c \]
را به دست می‌دهد.
وارون این رابطه نیز با همین روند ثابت می‌شود، تنها کافی است از 
\[ (a \backslash b) / c \rightarrow (a \backslash b) / c \]
بیاغازید.
\end{proof}

لم اخیر نشان می‌دهد  می‌توان به جای
$(a \backslash b) / c$
یا
$a \backslash (b / c)$
به ساده‌گی نوشت
$a \backslash b / c$.

\begin{lemma}
\label{lemma:1-3}
برای هر
$a, b, c, d \in \mathcal{T}$
اگر
$a \rightarrow c$
و
$b \rightarrow d$
آن‌گاه
\begin{enumerate}
\item
$a \otimes b \rightarrow c \otimes d$،
\item
$a / d \rightarrow c / b$،
\item
$d \backslash a \rightarrow b \backslash c$.
\end{enumerate}
\end{lemma}

برهان این لم در فصل سپسین داده می‌شود.
\par
چنان که در برهان لم 
\ref{lemma:1-2}
دیدید، ما از بازتابی بودن 
$\rightarrow$
و ویژه‌گی نخست
چونان \trans{بن‌داشت}{axiom} آغازیدیم، و از دیگر ویژه‌گی‌های جبر آیدکیویچ-بارهیلل چونان قاعده‌های استنتاج بهره بردیم.
می‌توان این دید را مانند حساب‌های منطقی صورت بست، تا هم برهان‌های درختی ساده‌تر پدید آیند، و هم بهتر بتوان این جبر را مطالعه کرد.
\section{حساب لمبک}

فرض کنید 
$\mathfrak{A} = (\mathcal{A}, \otimes, /, \backslash, \rightarrow)$
یک جبر آیدکیویچ-بارهیلل باشد، و
$\mathcal{T} = \langle \mathcal{A} \rangle$
مجموعه‌ی گونه‌ها باشد.
\begin{definition}
برای
$a, b, c \in \mathcal{T}$
حساب نحوی را با بن‌داشت‌های
\begin{enumerate}
\item
بازتاب 
$\rightarrow$
(ویژه‌گی 3-الف):
$$ a \rightarrow a, $$
\item
ویژه‌گی 1-الف:
$$ (a \otimes b) \otimes c \rightarrow a \otimes (b \otimes c), $$
\item
ویژه‌گی 1-ب:
$$ a \otimes (b \otimes c) \rightarrow (a \otimes b) \otimes c, $$
\end{enumerate}
و قاعده‌های استنتاجی 
\begin{enumerate}
\item
ترایایی 
$\rightarrow$
(ویژه‌گی 3-ب):
$$
\infer[(3)]{a \rightarrow c}{a \rightarrow b & b \rightarrow c}
$$
\item
ویژه‌گی 2:
$$
\begin{matrix}
\infer[(2)]{a \rightarrow c / b}{a \otimes b \rightarrow c}
\hspace{3em} & \hspace{3em}
\infer[(2)]{b \rightarrow a \backslash c}{a \otimes b \rightarrow c}
\\ \vspace{0.1em} \\
\infer[(2)]{a \otimes b \rightarrow c}{a \rightarrow c / b}
\hspace{3em} & \hspace{3em}
\infer[(2)]{a \otimes b \rightarrow c}{b \rightarrow a \backslash c}
\end{matrix}
$$
\end{enumerate}
تعریف کنید.
\end{definition}

به روشني یک رابطه در حساب لمبک اثبات‌پذیر است اگر و تنها اگر توسط ویژه‌گی‌های جبرهای آیدکیویچ-بارهیلل باشد.
اکنون ما به کمک حساب لمبک لم
\ref{lemma:1-3}
را ثابت می‌کنیم.

\begin{proof}
\begin{enumerate}
\item
$$
\infer[(3)]{a \otimes b \rightarrow c \otimes d}{
    \infer[(2)]{a \otimes b \rightarrow c \otimes b}{
        \infer[(3)]{a \rightarrow (c \otimes b) / b}{
            a \rightarrow c
            &
            \infer[(2)]{c \rightarrow (c \otimes b) / b}{c \otimes b \rightarrow c \otimes b}
        }
    }
    &
    \infer[(2)]{c \otimes b \rightarrow c \otimes d}{
        \infer[(3)]{b \rightarrow c \backslash (c \otimes d)}{
            b \rightarrow d
            &
            \infer[(2)]{d \rightarrow c \backslash (c \otimes d)}{c \otimes d \rightarrow c \otimes d}
        }
    }
}
$$
\item
$$
\infer[(3)]{a / d \rightarrow c / b}{
    \infer[(2)]{a / d \rightarrow a / b}{
        \infer[(2)]{(a / d) \otimes b \rightarrow a}{
            \infer[(3)]{b \rightarrow (a / d) \backslash a}{
                b \rightarrow d
                &
                \infer={d \rightarrow (a / d) \backslash a}{}
            }
        }
    }
    &
    \infer[(2)]{a / b \rightarrow c / b}{
        \infer[(3)]{(a / b) \otimes b \rightarrow c}{
            \infer={(a / b) \otimes b \rightarrow a}{}
            &
            a \rightarrow c
        }
    }
}
$$
\item
$$
\infer[(3)]{d \backslash a \rightarrow b \backslash c}{
    \infer[(2)]{d \backslash a \rightarrow d \backslash c}{
        \infer[(3)]{d \otimes (d \backslash a) \rightarrow c}{
            \infer={d \otimes (d \backslash a) \rightarrow a}{}
            &
            a \rightarrow c
        }
    }
	&
	\infer[(2)]{d \backslash c \rightarrow b \backslash c}{
        \infer[(2)]{b \otimes (d \backslash c) \rightarrow c}{
            \infer[(3)]{b \rightarrow c / (d \backslash c)}{
                b \rightarrow d
                &
                \infer={d \rightarrow c / (d \backslash c)}{}
            }
        }
    }
}
$$
که در برهان دوم ما از دو بخش نخست قضیه‌ی
\ref{theorem:1-1}
بهره جستیم تا درخت اثبات شلوغ نشود.
هم‌چنین در برهان سوم از صورت دیگر برافرازش گونه بهره بردیم. نگاه کنید به قضیه‌ی
\ref{theorem:1-1inverse}.
\end{enumerate}
\end{proof}

\begin{contract}
از این پس در حساب لمبک نام قاعده‌ی به‌کار‌رفته را نمی‌نویسیم، زیرا از تعداد گره‌های بالا و پایین آن مشخص می ‌شود.
هم‌چنین استنتاج‌های دوخطی به معنای به‌کارگیری لم‌ها و قضیه‌های ثابت‌شده اند.
\end{contract}

اکنون به یاری لم‌های ثابت‌شده بخش‌های دیگر قضیه‌ی
\ref{theorem:1-1}
را ثابت می‌کنیم.

\begin{proof}

\begin{enumerate}
\item[3.]
$$
\infer{(a/b)\otimes(b/c) \rightarrow a/c}{
    \infer{b/c \rightarrow (a/b)\backslash(a/c)}{
        \infer={b/c \rightarrow ((a / b) \backslash a) / c}{
            \infer={b \rightarrow (a/b)\backslash a}{}
            &
            c \rightarrow c
        }
        &
        \infer={((a/b) \backslash a)/c \rightarrow (a/b)\backslash(a/c)}{}
    }
}
$$
\item[4.]
قانون گیچ نتیجه‌ی مستقیم طرفین-وسطین است:
$$
\infer{a/b \rightarrow (a /c ) / (b / c)}{
    \infer={(a/b)\otimes(b/c) \rightarrow a/c}{}
}
$$
\item[5.]
قانون کری با یک بار به‌کارگیری ویژه‌گی 2 تبدیل می‌شود به
$((c/b)/a)\otimes(a \otimes b) \rightarrow c$
که با دو بار به‌کارگیری قانون آیدکیویچ به دست می‌آید.
\end{enumerate}
\end{proof}

قضیه‌ی 
\ref{theorem:1-1}
بر اساس عملگر 
$/$
نوشته شده بود. قانون‌هایي همانند برای عملگر
$\backslash$
نیز برقرار اند:

\begin{theorem}
\label{theorem:1-1inverse}
برای هر
$a, b, c \in \mathcal{T}$
رابطه‌های زیر درست اند:
\begin{enumerate}
\item
(قانون آیدوکیویچ)
$b \otimes (b \backslash a) \rightarrow a$،
\item 
(برافرازش گونه)
$b \rightarrow a / (b \backslash a)$،
\item
(طرفین-وسطین)
$(a \backslash b) \otimes (b \backslash c) \rightarrow a \backslash c$،
\item
(قانون گیچ)
$a \backslash b \rightarrow (c \backslash a) / (c \backslash b)$.
\item
(قانون کری)
$(a \otimes b) \backslash c \leftrightarrow b \backslash (a \backslash c)$.
\end{enumerate}
\end{theorem}

قانون کری به علت دوطرفه بودن ابزار بسیار قدرتمندي در اثبات قضیه‌های سپسین است.

s
\section{جبر دوخطی}

\trans{جبر دوخطی}{bilinear algebra}
مربوط است به
\trans[منطق‌های دوخطی]{منطق دوخطی}{bilinear logic}.
این اصطلاح را نخست لمبک معرفی کرد.
\cite{Lambek:Bilinear}

\begin{definition}
فرض کنید
$1 \in \mathcal{T}$
عضو خنثای ضرب در جبر آیدکیویچ-بارهیلل باشد. یعنی داریم
$$ 1 \otimes a \leftrightarrow a \leftrightarrow a \otimes 1. $$
در این صورت این جبر را یک‌دار می‌نامیم.
\end{definition}

\begin{lemma}
\label{lemma:1unique}
اگر 
$\mathfrak{A}$
یک‌دار باشد، یک آن یکتا ست.
\end{lemma}
\begin{proof}
داریم
$1 \leftrightarrow 1 \otimes 1' \leftrightarrow 1'$.
\end{proof}

\begin{proposition}
\label{proposition:element1-1}
برای 
$a \in \mathcal{A}$
داریم
$1 \rightarrow a / a$ 
و
$1 \rightarrow a \backslash a$.
\end{proposition}

\begin{theorem}
\label{lemma:element1-2}
برای
$a \in \mathcal{T}$
داریم
$a/1 \leftrightarrow a \leftrightarrow 1 \backslash a$.
\end{theorem}
\begin{proof}
از 
$a \otimes 1 \rightarrow a$
و
$1 \otimes a \rightarrow a$
به ترتیب داریم
$a \rightarrow a / 1$
و
$a \rightarrow 1 \backslash a$.
از طرفي داریم
\[ a/1 \rightarrow (a/1) \otimes 1 \rightarrow a, \]
و
\[ 1 \backslash a \rightarrow 1 \otimes (1 \backslash a) \rightarrow a. \]
\end{proof}

اگر 
$\mathfrak{A}$
یک‌دار باشد، تعریف کنید
$a_\ell \leftrightarrow 1 / a$
و
$a_r \leftrightarrow a \backslash 1$.
از قانون آدیوکیویچ نتیجه می‌شود
$a_\ell \otimes a \rightarrow 1$
و
$a \otimes a_r \rightarrow 1$.
عکس این دو رابطه درست نیست، در نتیجه نمی‌توان این دو عنصر را وارون‌های چپ و راست نامید.
برای رسیدن یافتن عنصرهایي بر حسب
$a$
که در رابطه‌یي شیبه به عکس رابطه‌های بالا صدق کند، جبرمان را اندکي بیش‌تر محدود می‌کنیم.

\begin{definition}
فرض کنید عنصر
$0 \in \mathcal{A}$
دارای ویژه‌گی
\[ 0 / (a \backslash 0) \leftrightarrow a \leftrightarrow (0 / a) \backslash 0 \]
باشد، تعریف می‌کنیم
\[ a^\ell \leftrightarrow 0 / a, \ \ \ \  a^r \leftrightarrow a \backslash 0. \]
\end{definition}

طبق تعریف 
$0$
داریم
$a^{r \ell} \leftrightarrow a \leftrightarrow a^{\ell r}$.
پس برای عنصر
$a \in \mathcal{T}$
از به کار بردن چندباری عملگرهای 
$\cdot^\ell$
و
$\cdot^r$
روی 
$a$
زنجیر
\[ \dots, a^{\ell \ell}, a^\ell, a, a^r, a^{rr}, \dots \]
پدید می‌آید.

\begin{proposition}
\label{proposition:order-reversing}
از 
$a \rightarrow b$
نتیجه می‌شود 
$b^r \rightarrow a^r$
و
$b^\ell \rightarrow a^\ell$.
به سخن دیگر این دو عمل ترتیب را وارونه می‌کنند.
\end{proposition}

\begin{proof}
اگر
$a \rightarrow b$،
از لم
\ref{lemma:1-3}
داریم
$0/b \rightarrow 0/a$،
یعنی
$b^\ell \rightarrow a^\ell$.
همانند این می‌توان نشان داد
$b^r \rightarrow a^r$.
یعنی این دو عمل ترتیب را وارون می‌کنند.
\end{proof}

\begin{theorem}
برای هر
$a, b \in \mathcal{T}$
داریم
$(a^\ell \otimes a^\ell)^r \leftrightarrow (a^r \otimes a^r)^\ell$.
\end{theorem}
\begin{proof}
برهان این قضیه از قانون کری استفاده می‌کند.
\begin{eqnarray}
(a^\ell \otimes b^\ell)^r &\leftrightarrow& ((0 / a) \otimes (0 / b)) \backslash 0 \nonumber \\
&\leftrightarrow& (0 / b) \backslash ((0 / a) \backslash 0) \nonumber \\
&\leftrightarrow& (0 / b) \backslash (0 / (a \backslash 0)) \nonumber \\
&\leftrightarrow& ((0 / b) \backslash 0) / (a \backslash 0) \nonumber \\
&\leftrightarrow& (0 / (b \backslash 0)) / (a \backslash 0) \nonumber \\
&\leftrightarrow& 0 / ((a \backslash 0) \otimes (b \backslash 0)) \nonumber \\
&\leftrightarrow& (a^r \otimes b^r)^\ell. \nonumber
\end{eqnarray}
\end{proof}

\begin{definition}
برای هر 
$a, b \in \mathcal{T}$
تعریف کنید
\[ a \oplus b \leftrightarrow (b^\ell \otimes a^\ell)^r \leftrightarrow (b^r \otimes a^r)^\ell. \]
\end{definition}

\begin{lemma}
داریم
\begin{enumerate}
\item
$1^\ell \leftrightarrow 0 \leftrightarrow 1^r$.
\item
$0^\ell \leftrightarrow 1 \leftrightarrow 0^r$.
\end{enumerate}
\end{lemma}

\begin{proof}
نخستین بخش نتیجه‌ی بی‌واسطه‌ی قضیه‌ی
\ref{proposition:element1-1}
است.
اکنون از بخش نخست داریم
$0 \leftrightarrow 1^\ell$.
از گزاره‌ی
\ref{proposition:order-reversing}
نتیجه می‌شود
$0^r \leftrightarrow 1^{\ell r} \leftrightarrow 1$.
به همین صورت از
$0 \leftrightarrow 1^r$.
داریم
$0^\ell \leftrightarrow 1^{r \ell} \leftrightarrow 1$.
\end{proof}

\begin{theorem}
عنصر 
$0$
عضو خنثای 
$\oplus$
است.
\end{theorem}
\begin{proof}
برای
$a \in \mathcal{T}$
داریم
\begin{eqnarray}
0 \oplus a &\leftrightarrow& (a^\ell \otimes 0^\ell)^r \nonumber \\
&\leftrightarrow& (a^\ell \otimes 1)^r \nonumber \\
&\leftrightarrow& (a^\ell)^r \nonumber \\
&\leftrightarrow& a. \nonumber
\end{eqnarray}
به همین ترتیب داریم 
$a \oplus 0 \leftrightarrow a$.
\end{proof}

بنا بر تعریف این دو عمل به ساده‌گی از قانون آدیوکیویچ نتیجه می‌شود 
$a^\ell \otimes a \rightarrow 0$
و
$a \otimes a^r \rightarrow 0$.
در رابطه‌ی نخست 
$a$
را به صورت 
$a^{r \ell}$
بنویسید، و عمل 
$\cdot^r$
را روی کل رابطه تأثیر دهید، رابطه‌ی 
$0^r \rightarrow (a^\ell \otimes a^{r\ell})^r$
به دست می‌آید، که همان
$1 \rightarrow a^r \oplus a$
است.
همانند همین داریم 
$1 \rightarrow a \oplus a^\ell$.
پس قضیه‌ی زیر را می‌توان نوشت.

\begin{theorem}
\label{theorem:3-compact}
برای 
$a, b, c \in \mathcal{T}$
داریم
\begin{enumerate}
\item
$a^\ell \otimes a \rightarrow 0$،
\item
$a \otimes a^r \rightarrow 0$،
\item
$1 \rightarrow a \oplus a^\ell$،
\item
$1 \rightarrow a^r \oplus a$.
\end{enumerate}
\end{theorem}

سپس‌تر در گرامر پیش‌گروهی  به این قضیه بازخواهیم گشت.

\begin{definition}
چندتایی
$\mathfrak{B} = (\mathcal{A}, \otimes, /, \backslash, \rightarrow)$
را یک جبر دوخطی نامیم هر گاه
$(\langle \mathcal{A} \rangle, \rightarrow)$
مجموعه‌ی مرتب جرئی و
$(\langle \mathcal{A} \rangle, \otimes, 1)$
یک 
\trans{تکواره}{monoid}
باشد که
$1 \in \mathcal{A}$،
عمل‌های
$/$
و
$\backslash$
دارای این ویژه‌گی باشند که
برای هر
$a, b, c \in \langle \mathcal{A} \rangle$،
\[
a \otimes b \rightarrow c
\ \ \ \ \textrm{اگر تنها و اگر}\ \ \ \ 
a \rightarrow c / b
\ \ \ \ \textrm{اگر تنها و اگر}\ \ \ \ 
b \rightarrow a \backslash c
\]
و عنصر 
$0 \in \mathcal{A}$
دارای ویژه‌گی
\begin{equation}
\label{eq:D}
0 / (a \backslash 0) \leftrightarrow a \leftrightarrow (0 / a) \backslash 0
\end{equation}
برای هر 
$a \in \langle \mathcal{A} \rangle$
باشد.
ویژه‌گی 
\ref{eq:D}
را ویژه‌گی 
\trans{دوگان‌سازی}{dualizing}
و
عنصر
$0$
را
\trans{دوگان‌ساز}{dualizer}
می‌نامند.
\end{definition}
\section{جبر دوخطی مشبکه‌یی}

ویژه‌گی سوم جبر آیدکیویچ-بارهیلل ایجاب می‌کند
عملگر
$\rightarrow$
یک رابطه‌ی ترتیب جزئی باشد. 
بنا بر این، می‌توان چهار نماد مشبکه‌یی را برای این عملگر تعریف کرد.
فرض کنید
$a \wedge b$
و
$a \vee b$
به ترتیب
\trans{زیرینه}{infimum}
و
\trans{زبرینه}{supremum}
$a$
و
$b$
را بنمایانند.
و نیز فرض کنید 
$\bot$
و
$\top$
\trans{زیر}{bottom}
و
\trans{زبر}{top}
باشند.
اکنون 
$\langle \mathcal{T} \rangle_{\mathrm{lattice}}$
را کوچک‌ترین مجموعه‌یي بگیرید که دارای زیر و زبر است،
$\mathcal{T}$
را در خود می‌گنجاند، و تحت زیرینه و زبرینه بسته است.

لمبک در 
\cite{Lambek:Bilinear}
اشاره کرده‌است که با افزایاندن سه قاعده‌ی ساختاری دستگاه
\trans{گنتسن}{Genzen}
به جبر دوخطی مشبکه‌یی، این جبر به حساب گنتسن بدل می‌شود.
قضیه‌ی زیر در
\cite{Lambek:Bilinear}
بدون اثبات آورده شده‌است، که منظور از این سخنان را بیان می‌کند.
ما سپس‌تر به دقت این گفته‌ی لمبک را اثبات خواهیم کرد، ولی نخست می‌کوشیم ارتباط نمادهای مشبکه‌یی را به جبر دوخطی مشبکه‌یی بدون این 3 قاعده‌ی ساختاری بیابیم.

\begin{theorem}
فرض کنید عمل 
$\otimes$
جابه‌جایی باشد، یعنی
$a \otimes b \rightarrow b \otimes a$،
و نیز بداریم
$a \rightarrow 1$
و
$a \rightarrow a \otimes a$،
آن‌گاه داریم
$1 \leftrightarrow \top$،
$a \otimes b \leftrightarrow a \wedge b$،
و
$c / b \leftrightarrow b \backslash c$.
\end{theorem}

\begin{proof}
نخست برای هر 
$a \in \langle \mathcal{T} \rangle_{\mathrm{lattice}}$
داریم
$a \rightarrow 1$
(بنا به فرض)
و
$a \rightarrow \top$
(بنا به مشبکه بودن)،
پس
$1 \leftrightarrow \top$.
از سوی دیگر
\[
\infer{a \otimes b \rightarrow a}{
	\infer={a \otimes b \rightarrow a \otimes 1}{
		\infer{a \rightarrow a}{}
		&
		b \rightarrow 1
	}
	&
	\infer={a \otimes 1 \rightarrow a}{}
}
\]
و به همین سان
$a \otimes b \rightarrow b$.
اکنون فرض کنید 
$c \rightarrow a$
و
$c \rightarrow b$،
داریم
\[
\infer{c \rightarrow a \otimes b}{
	c \rightarrow c \otimes c
	&
	\infer={c \otimes c \rightarrow a \otimes b}{
		c \rightarrow a
		&
		c \rightarrow b
	}
}
\]
پس داریم
$a \wedge b \leftrightarrow a \otimes b$.
سومین بخش از حکم نیز از ویژه‌گی آیدکیویچ و جابه‌جایی بودن عمل 
$\otimes$
نتیجه می‌شود.
\end{proof}

\begin{lemma}
\label{proposition:1iff0}
داریم
$1 \leftrightarrow \top$
اگر و تنها اگر
$0 \leftrightarrow \bot$.
\end{lemma}

\begin{proof}
فرض کنید 
$1 \leftrightarrow \top$،
پس برای هر 
$a \in \langle \mathcal{T} \rangle_{\mathrm{lattice}}$
داریم 
$a^\ell \rightarrow 1$،
پس داریم 
$0 \leftrightarrow 1^r \rightarrow a^{\ell r} \leftrightarrow a$.
یعنی
$0 \leftrightarrow \bot$.
اکنون به عکس فرض کنید
$0 \leftrightarrow \bot$،
پس برای هر 
$a \in \langle \mathcal{T} \rangle_{\mathrm{lattice}}$
داریم 
$0 \rightarrow a^\ell$،
پس داریم 
$a \leftrightarrow a^{\ell r} \rightarrow 0^r \leftrightarrow 1$.
یعنی
$1 \leftrightarrow \top$.
\end{proof}

\begin{definition}
در جبر دوخطی مشبکه‌یی، برای 
$a \in \langle \mathcal{T} \rangle_{\mathrm{lattice}}$
تعریف کنید
$\neg_r a \leftrightarrow a \backslash \bot$
و
$\neg_\ell a \leftrightarrow \bot / a$.
\end{definition}

\begin{lemma}
\label{lemma:1-neg-0}
داریم 
$\neg_r a \rightarrow a^r$
و
$\neg_\ell a \rightarrow a^\ell$.
\end{lemma}

\begin{proof}
داریم
$\bot \rightarrow 0$
و
$a \rightarrow a$،
پس از لم
\ref{lemma:1-3}
داریم
$\bot / a \rightarrow 0 / a$
و
$a \backslash \bot \rightarrow a \backslash 0$،
یعنی
$\neg_\ell a \rightarrow a^\ell$
و
$\neg_r a \rightarrow a^r$.
\end{proof}

\begin{theorem}
\label{theorem:4-0-bot-1-top}
داریم
$1 \leftrightarrow \top$
و
$0 \leftrightarrow \bot$.
\end{theorem}

\begin{proof}
توجه کنید که 
$\bot^r \leftrightarrow \bot \backslash 0 \leftrightarrow \neg_\ell 0$.
پس از لم
\ref{lemma:1-neg-0}
داریم
$\bot^r \leftrightarrow \neg_\ell 0 \rightarrow 0^\ell$.
از به‌کارگیری عمل
$\cdot^\ell$
روی دو سوی نابرابری داریم
$0^{\ell \ell} \rightarrow \bot^{r \ell} \leftrightarrow \bot$.
اکنون از
$0^\ell \rightarrow 1$
داریم
$1^\ell \rightarrow 0^{\ell \ell}$،
یعنی
$0 \rightarrow 0^{\ell \ell} \rightarrow \bot$.
پس 
$0 \rightarrow \bot$.
یعنی
$0 \leftrightarrow \bot$
و از لم
\ref{proposition:1iff0}
حکم نتیجه می‌شود.
\end{proof}

\begin{definition}
عنصر 
$c$
را چرخاننده می‌خوانیم هر گاه برای هر 
$a, b$
از
$a \otimes b \rightarrow c$
بتوان نتیجه گرفت
$b \otimes a \rightarrow c$.
اگر جبر دوخطی دارای عضوي چرخاننده باشد، آن را جبر
\trans{چرخشی}{cyclic}
خوانیم.
\end{definition}

\begin{proposition}
\label{proposition:cyclic-iff}
عنصر
$c$
چرخاننده است اگر و تنها اگر برای هر
$a$
بداریم
$c / a \leftrightarrow a \backslash c$.
\end{proposition}

به روشنی در جبر دوخطی مشبکه‌یی، 
$1$
یک چرخاننده است. هم‌چنین اگر عمل 
$\otimes$
جابه‌جایی باشد، هر عنصر جبرمان چرخاننده است.
از گزاره‌ی
\ref{proposition:cyclic-iff}
داریم 
$\bot$
چرخاننده است اگر و تنها اگر
$a^\ell \leftrightarrow a^r$
برای هر
$a \in \langle \mathcal{T} \rangle_{\mathrm{lattice}}$.

\begin{theorem}
در جبر دوخطی مشبکه‌یی داریم
$a^\ell \leftrightarrow a^r$
برای هر
$a \in \langle \mathcal{T} \rangle_{\mathrm{lattice}}$.
\end{theorem}

\begin{proof}
از قضیه‌ی
\ref{theorem:3-compact}
داریم
$a^\ell \otimes a \rightarrow 0$
و از قضیه‌ی
\ref{theorem:4-0-bot-1-top}
داریم
$0 \rightarrow a^\ell \otimes a$،
پس داریم
$a^\ell \otimes a \leftrightarrow 0$.
به همین صورت
$a \otimes a^r \leftrightarrow 0$.
داریم
\end{proof}
\section{جبر دوخطی فشرده}

\begin{definition}
فرض کنید
$\mathfrak{B}$
یک جبر دوخطی باشد.
$\mathfrak{B}$
را یک فشرده نامیم هر گاه 
$1 \leftrightarrow 0$
و
$\otimes = \oplus$.
\end{definition}

\begin{contract}
از این پس جبرهای دوخطی فشرده را با
$\mathfrak{K}$
نشان می‌دهیم.
و نیز عمل ضرب این جبر را نمی‌نویسیم یا با 
$\cdot$
نشان می‌دهیم، زیرا ابهامي به وجود نمی‌آید.
هم‌چنین اگر
$\mathfrak{K} = (\mathcal{A}, \otimes, /, \backslash, \rightarrow)$،
منظور از
$a \in \mathfrak{K}$
این است که
$a \in \langle \mathcal{A} \rangle$.
\end{contract}

\begin{theorem}
برای
$a \in \mathfrak{K}$
داریم
\[ a^\ell \cdot a \rightarrow 1 \rightarrow a \cdot a^\ell, \]
و
\[ a \cdot a^r \rightarrow 1 \rightarrow a^r \cdot a. \]
\end{theorem}


\appendix

\chapter{فهرست قضیه‌ها}

\backmatter

%\PrepareForBibliography

\setlength{\baselineskip}{0.8cm}

\latin

\bibliographystyle{unsrt}
\bibliography{resources/resources}

\persian


\PrepareForLatinPages
\date{April 2014}
\logo{\includegraphics[scale=.4]{img/logo-en}}
\title{\sffamily\enTitle}
% uncomment following lines only if you have defined commands for two-lines-title at the beginning of this file
\titlelineone{\enTitleLineOne}
\titlelinetwo{\enTitleLineTwo}
\author{\sffamily\enAuthor\vspace{-1em}}
\university{\normalfont\bfseries Sharif University of Technology\\Department of Mathematical Sciences}
\subject{Pure Mathematics}
\supervisor{\sffamily Prof. Dr. Muhammad Ardeshir\vspace{-1em}}
%If you don't have a consultant professor, comment following line
\consult{\sffamily Dr. Bahram Vazirnezhad\vspace{-1em}}
\begin{abstract}{\enKeywords}
\input{general/abstract-en}
\end{abstract}
\makethesistitle
\end{document}
