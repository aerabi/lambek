\section{جبر دوخطی}

\trans{جبر دوخطی}{bilinear algebra}
مربوط است به
\trans[منطق‌های دوخطی]{منطق دوخطی}{bilinear logic}.
این اصطلاح را نخست لمبک معرفی کرد.
\cite{Lambek:Bilinear}

\begin{definition}
فرض کنید
$1 \in \mathcal{T}$
عضو خنثای ضرب در جبر آیدکیویچ-بارهیلل باشد. یعنی داریم
$$ 1 \otimes a \leftrightarrow a \leftrightarrow a \otimes 1. $$
در این صورت این جبر را یک‌دار می‌نامیم.
\end{definition}

\begin{lemma}
\label{lemma:1unique}
اگر 
$\mathfrak{A}$
یک‌دار باشد، یک آن یکتا ست.
\end{lemma}
\begin{proof}
داریم
$1 \leftrightarrow 1 \otimes 1' \leftrightarrow 1'$.
\end{proof}

\begin{proposition}
\label{proposition:element1-1}
برای 
$a \in \mathcal{A}$
داریم
$1 \rightarrow a / a$ 
و
$1 \rightarrow a \backslash a$.
\end{proposition}

\begin{theorem}
\label{lemma:element1-2}
برای
$a \in \mathcal{T}$
داریم
$a/1 \leftrightarrow a \leftrightarrow 1 \backslash a$.
\end{theorem}
\begin{proof}
از 
$a \otimes 1 \rightarrow a$
و
$1 \otimes a \rightarrow a$
به ترتیب داریم
$a \rightarrow a / 1$
و
$a \rightarrow 1 \backslash a$.
از طرفي داریم
\[ a/1 \rightarrow (a/1) \otimes 1 \rightarrow a, \]
و
\[ 1 \backslash a \rightarrow 1 \otimes (1 \backslash a) \rightarrow a. \]
\end{proof}

اگر 
$\mathfrak{A}$
یک‌دار باشد، تعریف کنید
$a_\ell \leftrightarrow 1 / a$
و
$a_r \leftrightarrow a \backslash 1$.
از قانون آدیوکیویچ نتیجه می‌شود
$a_\ell \otimes a \rightarrow 1$
و
$a \otimes a_r \rightarrow 1$.
عکس این دو رابطه درست نیست، در نتیجه نمی‌توان این دو عنصر را وارون‌های چپ و راست نامید.
برای رسیدن یافتن عنصرهایي بر حسب
$a$
که در رابطه‌یي شیبه به عکس رابطه‌های بالا صدق کند، جبرمان را اندکي بیش‌تر محدود می‌کنیم.

\begin{definition}
فرض کنید عنصر
$0 \in \mathcal{A}$
دارای ویژه‌گی
\[ 0 / (a \backslash 0) \leftrightarrow a \leftrightarrow (0 / a) \backslash 0 \]
باشد، تعریف می‌کنیم
\[ a^\ell \leftrightarrow 0 / a, \ \ \ \  a^r \leftrightarrow a \backslash 0. \]
\end{definition}

طبق تعریف 
$0$
داریم
$a^{r \ell} \leftrightarrow a \leftrightarrow a^{\ell r}$.
پس برای عنصر
$a \in \mathcal{T}$
از به کار بردن چندباری عملگرهای 
$\cdot^\ell$
و
$\cdot^r$
روی 
$a$
زنجیر
\[ \dots, a^{\ell \ell}, a^\ell, a, a^r, a^{rr}, \dots \]
پدید می‌آید.

\begin{proposition}
\label{proposition:order-reversing}
از 
$a \rightarrow b$
نتیجه می‌شود 
$b^r \rightarrow a^r$
و
$b^\ell \rightarrow a^\ell$.
به سخن دیگر این دو عمل ترتیب را وارونه می‌کنند.
\end{proposition}

\begin{proof}
اگر
$a \rightarrow b$،
از لم
\ref{lemma:1-3}
داریم
$0/b \rightarrow 0/a$،
یعنی
$b^\ell \rightarrow a^\ell$.
همانند این می‌توان نشان داد
$b^r \rightarrow a^r$.
یعنی این دو عمل ترتیب را وارون می‌کنند.
\end{proof}

\begin{theorem}
برای هر
$a, b \in \mathcal{T}$
داریم
$(a^\ell \otimes a^\ell)^r \leftrightarrow (a^r \otimes a^r)^\ell$.
\end{theorem}
\begin{proof}
برهان این قضیه از قانون کری استفاده می‌کند.
\begin{eqnarray}
(a^\ell \otimes b^\ell)^r &\leftrightarrow& ((0 / a) \otimes (0 / b)) \backslash 0 \nonumber \\
&\leftrightarrow& (0 / b) \backslash ((0 / a) \backslash 0) \nonumber \\
&\leftrightarrow& (0 / b) \backslash (0 / (a \backslash 0)) \nonumber \\
&\leftrightarrow& ((0 / b) \backslash 0) / (a \backslash 0) \nonumber \\
&\leftrightarrow& (0 / (b \backslash 0)) / (a \backslash 0) \nonumber \\
&\leftrightarrow& 0 / ((a \backslash 0) \otimes (b \backslash 0)) \nonumber \\
&\leftrightarrow& (a^r \otimes b^r)^\ell. \nonumber
\end{eqnarray}
\end{proof}

\begin{definition}
برای هر 
$a, b \in \mathcal{T}$
تعریف کنید
\[ a \oplus b \leftrightarrow (b^\ell \otimes a^\ell)^r \leftrightarrow (b^r \otimes a^r)^\ell. \]
\end{definition}

\begin{lemma}
داریم
\begin{enumerate}
\item
$1^\ell \leftrightarrow 0 \leftrightarrow 1^r$.
\item
$0^\ell \leftrightarrow 1 \leftrightarrow 0^r$.
\end{enumerate}
\end{lemma}

\begin{proof}
نخستین بخش نتیجه‌ی بی‌واسطه‌ی قضیه‌ی
\ref{proposition:element1-1}
است.
اکنون از بخش نخست داریم
$0 \leftrightarrow 1^\ell$.
از گزاره‌ی
\ref{proposition:order-reversing}
نتیجه می‌شود
$0^r \leftrightarrow 1^{\ell r} \leftrightarrow 1$.
به همین صورت از
$0 \leftrightarrow 1^r$.
داریم
$0^\ell \leftrightarrow 1^{r \ell} \leftrightarrow 1$.
\end{proof}

\begin{theorem}
عنصر 
$0$
عضو خنثای 
$\oplus$
است.
\end{theorem}
\begin{proof}
برای
$a \in \mathcal{T}$
داریم
\begin{eqnarray}
0 \oplus a &\leftrightarrow& (a^\ell \otimes 0^\ell)^r \nonumber \\
&\leftrightarrow& (a^\ell \otimes 1)^r \nonumber \\
&\leftrightarrow& (a^\ell)^r \nonumber \\
&\leftrightarrow& a. \nonumber
\end{eqnarray}
به همین ترتیب داریم 
$a \oplus 0 \leftrightarrow a$.
\end{proof}

بنا بر تعریف این دو عمل به ساده‌گی از قانون آدیوکیویچ نتیجه می‌شود 
$a^\ell \otimes a \rightarrow 0$
و
$a \otimes a^r \rightarrow 0$.
در رابطه‌ی نخست 
$a$
را به صورت 
$a^{r \ell}$
بنویسید، و عمل 
$\cdot^r$
را روی کل رابطه تأثیر دهید، رابطه‌ی 
$0^r \rightarrow (a^\ell \otimes a^{r\ell})^r$
به دست می‌آید، که همان
$1 \rightarrow a^r \oplus a$
است.
همانند همین داریم 
$1 \rightarrow a \oplus a^\ell$.
پس قضیه‌ی زیر را می‌توان نوشت.

\begin{theorem}
\label{theorem:3-compact}
برای 
$a, b, c \in \mathcal{T}$
داریم
\begin{enumerate}
\item
$a^\ell \otimes a \rightarrow 0$،
\item
$a \otimes a^r \rightarrow 0$،
\item
$1 \rightarrow a \oplus a^\ell$،
\item
$1 \rightarrow a^r \oplus a$.
\end{enumerate}
\end{theorem}

سپس‌تر در گرامر پیش‌گروهی  به این قضیه بازخواهیم گشت.

\begin{definition}
چندتایی
$\mathfrak{B} = (\mathcal{A}, \otimes, /, \backslash, \rightarrow)$
را یک جبر دوخطی نامیم هر گاه
$(\langle \mathcal{A} \rangle, \rightarrow)$
مجموعه‌ی مرتب جرئی و
$(\langle \mathcal{A} \rangle, \otimes, 1)$
یک 
\trans{تکواره}{monoid}
باشد که
$1 \in \mathcal{A}$،
عمل‌های
$/$
و
$\backslash$
دارای این ویژه‌گی باشند که
برای هر
$a, b, c \in \langle \mathcal{A} \rangle$،
\[
a \otimes b \rightarrow c
\ \ \ \ \textrm{اگر تنها و اگر}\ \ \ \ 
a \rightarrow c / b
\ \ \ \ \textrm{اگر تنها و اگر}\ \ \ \ 
b \rightarrow a \backslash c
\]
و عنصر 
$0 \in \mathcal{A}$
دارای ویژه‌گی
\begin{equation}
\label{eq:D}
0 / (a \backslash 0) \leftrightarrow a \leftrightarrow (0 / a) \backslash 0
\end{equation}
برای هر 
$a \in \langle \mathcal{A} \rangle$
باشد.
ویژه‌گی 
\ref{eq:D}
را ویژه‌گی 
\trans{دوگان‌سازی}{dualizing}
و
عنصر
$0$
را
\trans{دوگان‌ساز}{dualizer}
می‌نامند.
\end{definition}