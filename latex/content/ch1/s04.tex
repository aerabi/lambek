\section{جبر دوخطی مشبکه‌یی}

ویژه‌گی سوم جبر آیدکیویچ-بارهیلل ایجاب می‌کند
عملگر
$\rightarrow$
یک رابطه‌ی ترتیب جزئی باشد. 
بنا بر این، می‌توان چهار نماد مشبکه‌یی را برای این عملگر تعریف کرد.
فرض کنید
$a \wedge b$
و
$a \vee b$
به ترتیب
\trans{زیرینه}{infimum}
و
\trans{زبرینه}{supremum}
$a$
و
$b$
را بنمایانند.
و نیز فرض کنید 
$\bot$
و
$\top$
\trans{زیر}{bottom}
و
\trans{زبر}{top}
باشند.
اکنون 
$\langle \mathcal{T} \rangle_{\mathrm{lattice}}$
را کوچک‌ترین مجموعه‌یي بگیرید که دارای زیر و زبر است،
$\mathcal{T}$
را در خود می‌گنجاند، و تحت زیرینه و زبرینه بسته است.

لمبک در 
\cite{Lambek:Bilinear}
اشاره کرده‌است که با افزایاندن سه قاعده‌ی ساختاری دستگاه
\trans{گنتسن}{Genzen}
به جبر دوخطی مشبکه‌یی، این جبر به حساب گنتسن بدل می‌شود.
قضیه‌ی زیر در
\cite{Lambek:Bilinear}
بدون اثبات آورده شده‌است، که منظور از این سخنان را بیان می‌کند.
ما سپس‌تر به دقت این گفته‌ی لمبک را اثبات خواهیم کرد، ولی نخست می‌کوشیم ارتباط نمادهای مشبکه‌یی را به جبر دوخطی مشبکه‌یی بدون این 3 قاعده‌ی ساختاری بیابیم.

\begin{theorem}
فرض کنید عمل 
$\otimes$
جابه‌جایی باشد، یعنی
$a \otimes b \rightarrow b \otimes a$،
و نیز بداریم
$a \rightarrow 1$
و
$a \rightarrow a \otimes a$،
آن‌گاه داریم
$1 \leftrightarrow \top$،
$a \otimes b \leftrightarrow a \wedge b$،
و
$c / b \leftrightarrow b \backslash c$.
\end{theorem}

\begin{proof}
نخست برای هر 
$a \in \langle \mathcal{T} \rangle_{\mathrm{lattice}}$
داریم
$a \rightarrow 1$
(بنا به فرض)
و
$a \rightarrow \top$
(بنا به مشبکه بودن)،
پس
$1 \leftrightarrow \top$.
از سوی دیگر
\[
\infer{a \otimes b \rightarrow a}{
	\infer={a \otimes b \rightarrow a \otimes 1}{
		\infer{a \rightarrow a}{}
		&
		b \rightarrow 1
	}
	&
	\infer={a \otimes 1 \rightarrow a}{}
}
\]
و به همین سان
$a \otimes b \rightarrow b$.
اکنون فرض کنید 
$c \rightarrow a$
و
$c \rightarrow b$،
داریم
\[
\infer{c \rightarrow a \otimes b}{
	c \rightarrow c \otimes c
	&
	\infer={c \otimes c \rightarrow a \otimes b}{
		c \rightarrow a
		&
		c \rightarrow b
	}
}
\]
پس داریم
$a \wedge b \leftrightarrow a \otimes b$.
سومین بخش از حکم نیز از ویژه‌گی آیدکیویچ و جابه‌جایی بودن عمل 
$\otimes$
نتیجه می‌شود.
\end{proof}

\begin{lemma}
\label{proposition:1iff0}
داریم
$1 \leftrightarrow \top$
اگر و تنها اگر
$0 \leftrightarrow \bot$.
\end{lemma}

\begin{proof}
فرض کنید 
$1 \leftrightarrow \top$،
پس برای هر 
$a \in \langle \mathcal{T} \rangle_{\mathrm{lattice}}$
داریم 
$a^\ell \rightarrow 1$،
پس داریم 
$0 \leftrightarrow 1^r \rightarrow a^{\ell r} \leftrightarrow a$.
یعنی
$0 \leftrightarrow \bot$.
اکنون به عکس فرض کنید
$0 \leftrightarrow \bot$،
پس برای هر 
$a \in \langle \mathcal{T} \rangle_{\mathrm{lattice}}$
داریم 
$0 \rightarrow a^\ell$،
پس داریم 
$a \leftrightarrow a^{\ell r} \rightarrow 0^r \leftrightarrow 1$.
یعنی
$1 \leftrightarrow \top$.
\end{proof}

\begin{definition}
در جبر دوخطی مشبکه‌یی، برای 
$a \in \langle \mathcal{T} \rangle_{\mathrm{lattice}}$
تعریف کنید
$\neg_r a \leftrightarrow a \backslash \bot$
و
$\neg_\ell a \leftrightarrow \bot / a$.
\end{definition}

\begin{lemma}
\label{lemma:1-neg-0}
داریم 
$\neg_r a \rightarrow a^r$
و
$\neg_\ell a \rightarrow a^\ell$.
\end{lemma}

\begin{proof}
داریم
$\bot \rightarrow 0$
و
$a \rightarrow a$،
پس از لم
\ref{lemma:1-3}
داریم
$\bot / a \rightarrow 0 / a$
و
$a \backslash \bot \rightarrow a \backslash 0$،
یعنی
$\neg_\ell a \rightarrow a^\ell$
و
$\neg_r a \rightarrow a^r$.
\end{proof}

\begin{theorem}
\label{theorem:4-0-bot-1-top}
داریم
$1 \leftrightarrow \top$
و
$0 \leftrightarrow \bot$.
\end{theorem}

\begin{proof}
توجه کنید که 
$\bot^r \leftrightarrow \bot \backslash 0 \leftrightarrow \neg_\ell 0$.
پس از لم
\ref{lemma:1-neg-0}
داریم
$\bot^r \leftrightarrow \neg_\ell 0 \rightarrow 0^\ell$.
از به‌کارگیری عمل
$\cdot^\ell$
روی دو سوی نابرابری داریم
$0^{\ell \ell} \rightarrow \bot^{r \ell} \leftrightarrow \bot$.
اکنون از
$0^\ell \rightarrow 1$
داریم
$1^\ell \rightarrow 0^{\ell \ell}$،
یعنی
$0 \rightarrow 0^{\ell \ell} \rightarrow \bot$.
پس 
$0 \rightarrow \bot$.
یعنی
$0 \leftrightarrow \bot$
و از لم
\ref{proposition:1iff0}
حکم نتیجه می‌شود.
\end{proof}

\begin{definition}
عنصر 
$c$
را چرخاننده می‌خوانیم هر گاه برای هر 
$a, b$
از
$a \otimes b \rightarrow c$
بتوان نتیجه گرفت
$b \otimes a \rightarrow c$.
اگر جبر دوخطی دارای عضوي چرخاننده باشد، آن را جبر
\trans{چرخشی}{cyclic}
خوانیم.
\end{definition}

\begin{proposition}
\label{proposition:cyclic-iff}
عنصر
$c$
چرخاننده است اگر و تنها اگر برای هر
$a$
بداریم
$c / a \leftrightarrow a \backslash c$.
\end{proposition}

به روشنی در جبر دوخطی مشبکه‌یی، 
$1$
یک چرخاننده است. هم‌چنین اگر عمل 
$\otimes$
جابه‌جایی باشد، هر عنصر جبرمان چرخاننده است.
از گزاره‌ی
\ref{proposition:cyclic-iff}
داریم 
$\bot$
چرخاننده است اگر و تنها اگر
$a^\ell \leftrightarrow a^r$
برای هر
$a \in \langle \mathcal{T} \rangle_{\mathrm{lattice}}$.

\begin{theorem}
در جبر دوخطی مشبکه‌یی داریم
$a^\ell \leftrightarrow a^r$
برای هر
$a \in \langle \mathcal{T} \rangle_{\mathrm{lattice}}$.
\end{theorem}

\begin{proof}
از قضیه‌ی
\ref{theorem:3-compact}
داریم
$a^\ell \otimes a \rightarrow 0$
و از قضیه‌ی
\ref{theorem:4-0-bot-1-top}
داریم
$0 \rightarrow a^\ell \otimes a$،
پس داریم
$a^\ell \otimes a \leftrightarrow 0$.
به همین صورت
$a \otimes a^r \leftrightarrow 0$.
داریم
\end{proof}