\section{حساب لمبک}

فرض کنید 
$\mathfrak{A} = (\mathcal{A}, \otimes, /, \backslash, \rightarrow)$
یک جبر آیدکیویچ-بارهیلل باشد، و
$\mathcal{T} = \langle \mathcal{A} \rangle$
مجموعه‌ی گونه‌ها باشد.
\begin{definition}
برای
$a, b, c \in \mathcal{T}$
حساب نحوی را با بن‌داشت‌های
\begin{enumerate}
\item
بازتاب 
$\rightarrow$
(ویژه‌گی 3-الف):
$$ a \rightarrow a, $$
\item
ویژه‌گی 1-الف:
$$ (a \otimes b) \otimes c \rightarrow a \otimes (b \otimes c), $$
\item
ویژه‌گی 1-ب:
$$ a \otimes (b \otimes c) \rightarrow (a \otimes b) \otimes c, $$
\end{enumerate}
و قاعده‌های استنتاجی 
\begin{enumerate}
\item
ترایایی 
$\rightarrow$
(ویژه‌گی 3-ب):
$$
\infer[(3)]{a \rightarrow c}{a \rightarrow b & b \rightarrow c}
$$
\item
ویژه‌گی 2:
$$
\begin{matrix}
\infer[(2)]{a \rightarrow c / b}{a \otimes b \rightarrow c}
\hspace{3em} & \hspace{3em}
\infer[(2)]{b \rightarrow a \backslash c}{a \otimes b \rightarrow c}
\\ \vspace{0.1em} \\
\infer[(2)]{a \otimes b \rightarrow c}{a \rightarrow c / b}
\hspace{3em} & \hspace{3em}
\infer[(2)]{a \otimes b \rightarrow c}{b \rightarrow a \backslash c}
\end{matrix}
$$
\end{enumerate}
تعریف کنید.
\end{definition}

به روشني یک رابطه در حساب لمبک اثبات‌پذیر است اگر و تنها اگر توسط ویژه‌گی‌های جبرهای آیدکیویچ-بارهیلل باشد.
اکنون ما به کمک حساب لمبک لم
\ref{lemma:1-3}
را ثابت می‌کنیم.

\begin{proof}
\begin{enumerate}
\item
$$
\infer[(3)]{a \otimes b \rightarrow c \otimes d}{
    \infer[(2)]{a \otimes b \rightarrow c \otimes b}{
        \infer[(3)]{a \rightarrow (c \otimes b) / b}{
            a \rightarrow c
            &
            \infer[(2)]{c \rightarrow (c \otimes b) / b}{c \otimes b \rightarrow c \otimes b}
        }
    }
    &
    \infer[(2)]{c \otimes b \rightarrow c \otimes d}{
        \infer[(3)]{b \rightarrow c \backslash (c \otimes d)}{
            b \rightarrow d
            &
            \infer[(2)]{d \rightarrow c \backslash (c \otimes d)}{c \otimes d \rightarrow c \otimes d}
        }
    }
}
$$
\item
$$
\infer[(3)]{a / d \rightarrow c / b}{
    \infer[(2)]{a / d \rightarrow a / b}{
        \infer[(2)]{(a / d) \otimes b \rightarrow a}{
            \infer[(3)]{b \rightarrow (a / d) \backslash a}{
                b \rightarrow d
                &
                \infer={d \rightarrow (a / d) \backslash a}{}
            }
        }
    }
    &
    \infer[(2)]{a / b \rightarrow c / b}{
        \infer[(3)]{(a / b) \otimes b \rightarrow c}{
            \infer={(a / b) \otimes b \rightarrow a}{}
            &
            a \rightarrow c
        }
    }
}
$$
\item
$$
\infer[(3)]{d \backslash a \rightarrow b \backslash c}{
    \infer[(2)]{d \backslash a \rightarrow d \backslash c}{
        \infer[(3)]{d \otimes (d \backslash a) \rightarrow c}{
            \infer={d \otimes (d \backslash a) \rightarrow a}{}
            &
            a \rightarrow c
        }
    }
	&
	\infer[(2)]{d \backslash c \rightarrow b \backslash c}{
        \infer[(2)]{b \otimes (d \backslash c) \rightarrow c}{
            \infer[(3)]{b \rightarrow c / (d \backslash c)}{
                b \rightarrow d
                &
                \infer={d \rightarrow c / (d \backslash c)}{}
            }
        }
    }
}
$$
که در برهان دوم ما از دو بخش نخست قضیه‌ی
\ref{theorem:1-1}
بهره جستیم تا درخت اثبات شلوغ نشود.
هم‌چنین در برهان سوم از صورت دیگر برافرازش گونه بهره بردیم. نگاه کنید به قضیه‌ی
\ref{theorem:1-1inverse}.
\end{enumerate}
\end{proof}

\begin{contract}
از این پس در حساب لمبک نام قاعده‌ی به‌کار‌رفته را نمی‌نویسیم، زیرا از تعداد گره‌های بالا و پایین آن مشخص می ‌شود.
هم‌چنین استنتاج‌های دوخطی به معنای به‌کارگیری لم‌ها و قضیه‌های ثابت‌شده اند.
\end{contract}

اکنون به یاری لم‌های ثابت‌شده بخش‌های دیگر قضیه‌ی
\ref{theorem:1-1}
را ثابت می‌کنیم.

\begin{proof}

\begin{enumerate}
\item[3.]
$$
\infer{(a/b)\otimes(b/c) \rightarrow a/c}{
    \infer{b/c \rightarrow (a/b)\backslash(a/c)}{
        \infer={b/c \rightarrow ((a / b) \backslash a) / c}{
            \infer={b \rightarrow (a/b)\backslash a}{}
            &
            c \rightarrow c
        }
        &
        \infer={((a/b) \backslash a)/c \rightarrow (a/b)\backslash(a/c)}{}
    }
}
$$
\item[4.]
قانون گیچ نتیجه‌ی مستقیم طرفین-وسطین است:
$$
\infer{a/b \rightarrow (a /c ) / (b / c)}{
    \infer={(a/b)\otimes(b/c) \rightarrow a/c}{}
}
$$
\item[5.]
قانون کری با یک بار به‌کارگیری ویژه‌گی 2 تبدیل می‌شود به
$((c/b)/a)\otimes(a \otimes b) \rightarrow c$
که با دو بار به‌کارگیری قانون آیدکیویچ به دست می‌آید.
\end{enumerate}
\end{proof}

قضیه‌ی 
\ref{theorem:1-1}
بر اساس عملگر 
$/$
نوشته شده بود. قانون‌هایي همانند برای عملگر
$\backslash$
نیز برقرار اند:

\begin{theorem}
\label{theorem:1-1inverse}
برای هر
$a, b, c \in \mathcal{T}$
رابطه‌های زیر درست اند:
\begin{enumerate}
\item
(قانون آیدوکیویچ)
$b \otimes (b \backslash a) \rightarrow a$،
\item 
(برافرازش گونه)
$b \rightarrow a / (b \backslash a)$،
\item
(طرفین-وسطین)
$(a \backslash b) \otimes (b \backslash c) \rightarrow a \backslash c$،
\item
(قانون گیچ)
$a \backslash b \rightarrow (c \backslash a) / (c \backslash b)$.
\item
(قانون کری)
$(a \otimes b) \backslash c \leftrightarrow b \backslash (a \backslash c)$.
\end{enumerate}
\end{theorem}

قانون کری به علت دوطرفه بودن ابزار بسیار قدرتمندي در اثبات قضیه‌های سپسین است.

s