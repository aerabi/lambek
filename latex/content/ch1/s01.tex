\section{جبر آیدوکیویچ-بارهیلل}

چندتایی 
$\mathfrak{A} = (\mathcal{A}, \otimes, /, \backslash, \rightarrow)$
را در نظر گیرید. مجموعه‌ی 
$\mathcal{T}$
را کوچک‌ترین مجموعه‌یي بگیرید که 
$\mathcal{A}$
را در بر دارد، و تحت سه عمل 
$\otimes, /, \backslash$
بسته است. این سه عمل در واقع صوری اند، و عنصرهای مجموعه‌ی
$\mathcal{T}$
به طور نحوی به استقرا ساخته می‌شوند.
مجموعه‌ی 
$\mathcal{T}$
را با
$\langle \mathcal{A} \rangle$
نیز نشان می‌دهند.
مجموعه‌ی
$\mathcal{T}$
را \trans{مجموعه‌ی گونه‌ها}{set of types}، و هر 
$t \in \mathcal{T}$
را یک \trans{گونه}{type} می‌نامیم.
مجموعه‌ی
$\mathcal{A}$
را نیز مجموعه‌ی گونه‌های \trans{ناشکستنی}{atomic} می‌نامیم.

\begin{definition}
چندتایی
$\mathfrak{A}$
را جبر آیدوکیویچ-بارهیلل می‌نامیم هر گاه

\begin{enumerate}
\item
برای هر
$a, b, c \in \mathcal{T}$
بداریم
$(a \otimes b) \otimes c \leftrightarrow a \otimes (b \otimes c)$،

\item 
برای هر
$a, b, c \in \mathcal{T}$
بداریم
$a \otimes b \rightarrow c$
اگر و تنها اگر
$a \rightarrow c/b$
اگر و تنها اگر
$b \rightarrow a \backslash c$،
\item
عمل 
$\rightarrow$
بازتابی و ترایا باشد.
\end{enumerate}
\end{definition}

در تعریف بالا،
$a \leftrightarrow b$
معادل است با
$a \rightarrow b$
و
$b \rightarrow a$.

\begin{theorem}
\label{theorem:1-1}
برای هر
$a, b, c \in \mathcal{T}$
رابطه‌های زیر درست اند:
\begin{enumerate}
\item
(قانون آیدوکیویچ)
$(a / b) \otimes b \rightarrow a$،
\item 
(برافرازش گونه)
$b \rightarrow (a / b) \backslash a$،
\item
(طرفین-وسطین)
$(a / b) \otimes (b / c) \rightarrow a / c$،
\item
(قانون \name{گیچ}{Geach})
$a / b \rightarrow (a / c) / (b / c)$،
\item
(قانون \name{کری}{Curry})
$(c / b) / a \leftrightarrow c / (a \otimes b)$.
\end{enumerate}
\end{theorem}

\begin{proof}
از بازتابی بودن
$\rightarrow$
داریم
$a / b \rightarrow a / b$
و از ویژه‌گی دوم جبر آیدوکیویچ-بارهیلل داریم
$(a / b) \rightarrow a / b$،
اگر و تنها اگر
$(a / b) \otimes b \rightarrow a$.
با استفاده از برابر هم‌ارز دیگر در ویژه‌گی دوم نیز به
$b \rightarrow (a / b) \backslash a$
می‌رسیم.
پس دو رابطه‌ی نخست به همین ساده‌گی برقرار اند. 
\end{proof}

برای اثبات دیگر بخش‌ها نخست چند لم نیاز است.

\begin{lemma}
\label{lemma:1-2}
داریم 
$(a \backslash b) / c \leftrightarrow a \backslash (b / c)$.
\end{lemma}

\begin{proof}
چنان که انتظار داریم، در اثبات این لم باید از ویژه‌گی نخست جبرهای آیدکیویچ-بارهیلل استفاده کنیم.
پس استراتژی برهان ترادیساندن این دو عملگر به عملگرهای 
$\otimes$
است، و سپس بهره‌جویی از ویژه‌گی نخست. این کار به ساده‌گی انجام می‌پذیرد. \\
بنا بر بازتابی بودن 
$\rightarrow$
داریم
\[ a \backslash (b / c) \rightarrow a \backslash (b / c), \]
پس با شکستن عمل سمت راست داریم
\[ a \otimes (a \backslash (b / c)) \rightarrow b / c, \]
و با بار دیگر شکستن عمل سمت راست و ترادیساندن آن به 
$\otimes$
داریم
\[ (a \otimes (a \backslash (b / c))) \otimes c \rightarrow b. \]
اکنون از ویژه‌گی نخست داریم
\[ a \otimes ((a \backslash (b / c)) \otimes c) \rightarrow (a \otimes (a \backslash (b / c))) \otimes c, \]
و از ترایا بودن 
$\rightarrow$
و آمیختن دو رابطه‌ی پیشین به دست می‌آوریم
\[ a \otimes ((a \backslash (b / c)) \otimes c) \rightarrow b, \]
با ترادیساندن 
$\otimes$
بیرونی به 
$\backslash$
می‌رسیم به
\[ (a \backslash (b / c)) \otimes c \rightarrow a \backslash b. \]
ترادیساندن عمل
$\otimes$
بیرونی این رابطه به 
$/$
رابطه‌ی 
\[ a \backslash (b / c) \rightarrow (a \backslash b) / c \]
را به دست می‌دهد.
وارون این رابطه نیز با همین روند ثابت می‌شود، تنها کافی است از 
\[ (a \backslash b) / c \rightarrow (a \backslash b) / c \]
بیاغازید.
\end{proof}

لم اخیر نشان می‌دهد  می‌توان به جای
$(a \backslash b) / c$
یا
$a \backslash (b / c)$
به ساده‌گی نوشت
$a \backslash b / c$.

\begin{lemma}
\label{lemma:1-3}
برای هر
$a, b, c, d \in \mathcal{T}$
اگر
$a \rightarrow c$
و
$b \rightarrow d$
آن‌گاه
\begin{enumerate}
\item
$a \otimes b \rightarrow c \otimes d$،
\item
$a / d \rightarrow c / b$،
\item
$d \backslash a \rightarrow b \backslash c$.
\end{enumerate}
\end{lemma}

برهان این لم در فصل سپسین داده می‌شود.
\par
چنان که در برهان لم 
\ref{lemma:1-2}
دیدید، ما از بازتابی بودن 
$\rightarrow$
و ویژه‌گی نخست
چونان \trans{بن‌داشت}{axiom} آغازیدیم، و از دیگر ویژه‌گی‌های جبر آیدکیویچ-بارهیلل چونان قاعده‌های استنتاج بهره بردیم.
می‌توان این دید را مانند حساب‌های منطقی صورت بست، تا هم برهان‌های درختی ساده‌تر پدید آیند، و هم بهتر بتوان این جبر را مطالعه کرد.