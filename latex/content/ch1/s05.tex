\section{جبر دوخطی فشرده}

\begin{definition}
فرض کنید
$\mathfrak{B}$
یک جبر دوخطی باشد.
$\mathfrak{B}$
را یک فشرده نامیم هر گاه 
$1 \leftrightarrow 0$
و
$\otimes = \oplus$.
\end{definition}

\begin{contract}
از این پس جبرهای دوخطی فشرده را با
$\mathfrak{K}$
نشان می‌دهیم.
و نیز عمل ضرب این جبر را نمی‌نویسیم یا با 
$\cdot$
نشان می‌دهیم، زیرا ابهامي به وجود نمی‌آید.
هم‌چنین اگر
$\mathfrak{K} = (\mathcal{A}, \otimes, /, \backslash, \rightarrow)$،
منظور از
$a \in \mathfrak{K}$
این است که
$a \in \langle \mathcal{A} \rangle$.
\end{contract}

\begin{theorem}
برای
$a \in \mathfrak{K}$
داریم
\[ a^\ell \cdot a \rightarrow 1 \rightarrow a \cdot a^\ell, \]
و
\[ a \cdot a^r \rightarrow 1 \rightarrow a^r \cdot a. \]
\end{theorem}